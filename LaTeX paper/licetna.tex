\documentclass[12pt]{article}
\usepackage[T1]{fontenc}
\usepackage{indentfirst}
\usepackage{anyfontsize}
\usepackage{setspace}
\usepackage{titlesec}
\usepackage[romanian]{babel}
%\usepackage{helvet}
%\renewcommand{\familydefault}{\sfdefault}

\usepackage{geometry}
 \geometry{
 a4paper,
 total={210mm,297mm},
 left=20mm,
 right=20mm,
 top=25mm,
 bottom=20mm,
 }
\setlength\parindent{15mm}

\begin{document}

\fontsize{20pt}{18pt}\selectfont
\title{\textbf{Microprocesorul RISC-V}}
\date{}
\maketitle

\vspace*{150mm}


\begingroup
    \fontsize{14pt}{12pt}\selectfont
	\textbf{Canditat: Dan-Alexandru Bulzan}
	\bigbreak
	\textbf{Coordonator științific: Ș.l.dr.ing Eugen-Horațiu Gurban}
\endgroup


\vspace*{\fill}
\begin{center}
\fontsize{14pt}{12pt}\selectfont
Sesiune: Iunie 2024
\end{center}


\newpage
\section{\centering INTRODUCERE}
\bigbreak
\subsection{SCOPUL ȘI MOTIVAȚIA LUCRĂRII}
\setstretch{1.2}

Implementarea personală a setului de instrucțiuni RISC-V, s-a născut din dorința de a realiza ceea ce poate fi considerat nimic mai puțin decât un apogeu al metodelor științifice din ultimul secol, și anume, procesorul.

Aceste dispozitive electronice reprezintă fundamentul tuturor științelor informatice, grație capacității computaționale intrinsece. Procesoarele, indiferent de gradul lor de specializare, au fost și rămân nucleul unei revoluții tehnologice pe care nici din pură ignoranță nu o putem omite, aceasta fiind prezentă până și în cele mai mundane aspecte ale vieții cotidiene.
Scopul acestei lucrări este de a traversa universul digital, începând din rădăcinile sale analogice, ajungând într-un final la organizarea ierarhică a numeroaselor entități digitale în a căror întregime se constituie un sistem de calcul complet funcțional.

Adesea este usor sa ne pierdem în complexitățiile ascunse printre miile de porți logice, un veritabil microcosm digital, însă prin mijloacele abstractizării și modularizării, proiectarea unui procesor devine nimic mai mult decât o modelare regulată a unui sistem descriptibil de operațiile algebrei Booleane. Pe parcursul lucrării, se va prezenta de asemenea o simplă implementare didactică a modulului de memorie cache, un component digital de o importanță deosebită, precum si problematica care cere o astfel de soluție.

Implementarea va fi realizată în limbajul de descriere hardware VHDL, entitățile urmând să fie simulate prin intermediul Vivado, soluție de design și sinteză hardware oferită de Advanced Micro Devices.

\begin{center}
\vspace*{40mm}
DE CONTINUAT, ALTERAT
\end{center}

\newpage
\section{\centering STUDIU BIBLIOGRAFIC}
\bigbreak
\subsection{ARHITECTURA RISC}
Înainte de realizarea unei analize asupra stadiului de dezvoltare și implementare al setului de instrucțiuni RISC-V, întelegerea locului pe care filozofia RISC o are în disciplina arhitecturii calculatoarelor, este de o importantă deosebită.

Acronimul RISC, face referintă la \textit{reduced instruction set computer} sau calculator cu set de instrucțiuni reduse. Un microprocesor care implementează o astfel de filozofie, utilizează un set de instructiuni compact și puternic optimizat, garantând execuția rapidă a fiecărei instrucțiuni. Prin urmare, o caracteristică a acestei abordări, este faptul că microprocesorul va fi nevoit să execute un numar mai ridicat de instrucțiuni pentru a realiza aceleași operații efectuate de un calculator cu set de instrucțiuni complex, cunoscut și sub acronimul de \textit{CISC}, printr-un număr observabil mai redus de instrucțiuni.

De-a lungul timpului, începând cu întemeierea arhitecturii RISC, au fost conceput mai multe seturi de instrucțiuni relevante, printre acestea enumerându-se următoarele: MIPS, ARM cât și setul care va reprezenta arhitectura procesorului implementat pe decursul acestei lucrari, RISC-V.

\subsection{FAMILIA SETURILOR DE INSTRUCȚIUNI ARM}
Seturile de instrucțiuni care aparțin familiei ARM sunt fără echivoc cele mai de succes dintre toate seturile aferente arhitecturii RISC. Acest succes este în mare parte datorat costurilor reduse de producție cât și eficienței computaționale ridicate. Dispozitivele dezvoltate în jurul microprocesoarelor ARM au un grade de utilitate ridicat, prezența acestora facându-se simțită într-o vastă gamă de domenii. Cele mai evidente utilizări sunt reprezentate de telefoanele mobile și computerele personale, însă arhitectura ARM a reușit să se etaleze până și în domeniul computerelor de înaltă performanță, prin intermediul supercomputerului Fugaku.

Arhitectura ARM s-a bucurat de decenii întregi de dezvoltare și prin urmare de vaste îmbunătățiri, ajungând la un grad înalt de maturitate, lucru care-i definește utilitate contemporană.

\subsection{SETUL DE INSTRUCȚIUNI RISC-V}
Setul de instrucțiuni RISC-V reprezintă una dintre cele mai noi adiții aduse mulțimii familiilor arhitecturii RISC. Acest ISA nu funcționează pe baza unei licențe de utilizare, fiind un standard deschis, este permisă folosirea sa tuturor entităților legale sau persoanelor care doresc implementarea unui microprocesor sau a unui sistem integrat bazându-se pe acest set.

\subsection{IMPLEMENTĂRI  RISC-V}
Datorită proliferării lipsite de licentă cât și împărțirii setului în extensii, se poate observa un constant flux de implementări, variind de la simple exemple didactice la sisteme cu module multicip complexe. Numeroase programe de studii care au ca scop dezvoltarea cunoștiintelor despre organizarea calculatoarelor, obișnuiesc să prezinte ca suport didactic implementări succinte ale unui nucleu RISC-V. Fiecare asemeni implementare prezintă ușoare diferențe arhitectural-organizatorice față de omologi săi. Aceste diferențe sunt produsul faptului că arhitectura RISC-V nu îngrădește utilizatorii săi într-o specifică topologie de organizare a modulelor care constituie în intregimea lor un microprocesor. Fiecare utilizator are astfel liber arbitru în definirea propriei organizări, atât timp cât respectă setul de instrucțiuni.

Se disting astfel două mari tipuri de microprocesoare RISC-V, ale căror implementări sunt disponibile spre analiză. Prima și cea mai comună este reprezentată de microprocesorul RISC-V SCP sau \textit{single cycle processor}, cea de a doua purtând numele de \textit{multi-cycle processor} sau pe scurt, MCP.


\begin{center}
\vspace*{60mm}
DE CONTINUAT
\end{center}

\newpage
\section{\centering FUNDAMENTARE TEORETICĂ}
\bigbreak
\subsection{GESTIONAREA COMPLEXITĂȚII}
Cand vine vorba de modelara unui sistem computațional de o complexitate ridicată, este de preferat să avem anumite fundamente în implementare, pe care să ne putem baza fără echivoc. În lipsa acestor principii este adesea usor să ne pierdem în complexitatea sistemului, rezultând astfel posibile erori care-și vor face simțită prezența în produsul final.

\subsubsection{ABSTRACTIZARE}
Abstractizarea este opusul specificității. Din punct de vedere conceptual, actul de abstractizare, indiferent de suportul teoretic asupra căruia este aplicat, ajută la simplificarea unei probleme a cărei complexități ar fi de altfel prea greu de tratat. Prin abstractizare, detalile de la un anumit nivel logic al unui sistem, sunt redate sumar și considerate ca atare de catre nivelele logice superioare.


Acest lucru poate fi observat într-o multitudine de domenii, de la arhitectura calculatoarelor la studiul fiziologiei medicale. De exemplu, bazându-ne pe cel din urmă domeniu enumerat, modul de funcționare a unui organism viu poate fi privit din mai multe perspective de abstractizare, începând de la interacțiunile biochimice si biomecanice de la nivelul unei celule, trecând pe urmă la modul în care aceste celule interacționeaza între ele formând variate țesuturi, ajungând într-un final la nivelul de abstracție al țesuturilor care împreună formează organe, fiecare nivel implicându-l direct pe precedentul său.

\subsubsection{MODULARITATE}
Modularizarea definește modul în care un sistem computațional va fi divizat în numeroase parți de sine stătătoare, acum numite module, fiecare cu un rol și o interfață de utilizare concis definită. Aceste module permit astfel reutilizarea  entităților pe care le definesc, ne mai fiind nevoie de irosirea unei perioade mari de timp cu diverse noi implementării care sunt congruente cu un modul deja existent. Modularizarea ne permite de asemenea înlocuirea unor părți ale sistemului nostru cu altele de o eficiență mai ridicată, cât timp acestea respectă aceeași interfață pentru a permite comunicarea cu modulele adiacente.

\subsubsection{IERARHIZARE}
Ierarhizarea implică ordonarea într-o arhitectură a modulelor anterior definite. Arhitectura, în cazul nostru, va fi reprezentată de modul de organizare a microprocesorului ce urmează a fi dezvoltat, microarhitectura acestuia. Organizarea ierarhică implică modularitatea dar vice-versa nu este mereu valabilă, modulele putând exista pe același nivel ierarhic, nefiind, prin urmare, subordonate unul altuia.

\subsection{ABSTRACȚIA NUMERICĂ}
Pentru a produce un rezultat de o oarecare utilitate, sistemele computaționale au nevoie de date. Aceste date sunt complet irelevante cât timp nu respectă un mod de reprezentare util sistemului. De asemenea, este importat de luat in considerare faptul ca datele hrănite pot avea semnificații diverse, complet obtuze una față de cealaltă.


Problema reprezentării datelor primește o importanță specială, deosebită chiar, dând naștere următoarei multitudini de întrebări, \textit{care este este modul corect de reprezentare}; \textit{cum asigurăm coerența datelor cu analizarea acestora de către sistemul de calcul}; \textit{cum ne asigurăm ca datele indiferent formatului ligibil uman, nu sunt iligibile procesorului.}


Pentru a răspunde pe deplin, trebuie mai întâi să definim tipul datelor pe care microprocesorul le va accepta. Este rapid evident, din natura sistemului, că datele vor trebui să fie numerice. Însă, nu la fel de evident este modul în care aceste numere vor fi reprezentate pentru a suporta toate operațiile admisibile unui motor logic-aritmetic.
\bigbreak
\end{document} 